In this section, we give more details about our organization, including how we will make decision and resolve conflicts. 

\subsection{Roles}

Every team member has two types of tasks: core tasks (e.g. programming goals) and administrative tasks (e.g. documentation and updated). Additionally, the coordinator is the official team communicator. However, it is clarified that this role does not imply a position of leadership or decision making (the decisions are jointly made by following the decision making and conflict resolution methodology in Section 2.4). The specific roles were already stated in Figure ~\ref{fig:Tasks}.

\subsection{Tools for Communication and Collaboration}

\begin{itemize}
	\item Slack (for communication purposes and specific tasks).
	\item GitHub (for code sharing and version control).
	\item Compulsory weekly meeting every Monday (12:00 PM - 2:00 PM).
	\item Compulsory Google Hangout every Sunday (5:00 PM- 6:00 PM).
	\item Extra Google hangout meeting (when needed).
	\item Meeting on Wednesdays 2:00 PM- 5:00 PM (if needed). 
	\item A timetable and advance curve control (i.e. where we are vs where we ought to be).
	\item Our decision-making and conflict resolution methodology (Section 2.4).
	\item Learning meeting (each 15th day we discuss what we can improve as a team)
\end{itemize}




\subsection{Peer Assessment Criteria}

Our assessment methodology is based on the following set of criteria.

\begin{table}[ht]
		\caption{Assessment Criteria}
		\label{tab:requirements}
    \begin{tabular}[r]{ | p{2.2cm} | p{13.3cm}|}
		\hline
		\centering\textbf{Criteria} & \centering\textbf{Description} &
		\hline
		Amount of work & Amount of work that each member contributes (e.g. difficulty or complexity in the sub-set of requirements achieved by the member). \\
    \hline
		Team work skills & i) communication skills, ii) cooperation (e.g. how a member supports others) and willingness to resolve conflicts or reach agreements.\\
    \hline
		Proactiveness & Self-motivation to explore and propose new alternatives or options to overcome any difficulty. These include initiative, willingness to learn, good attitude to explore new things, enthusiastic, creativity and innovation.\\
    \hline
    \end{tabular}
\end{table}

Each member evaluates others on a scale of 1 and 5 (where 1 is the lowest and 5 is the highest score). For each criterion, each member would write a little justification, which could be replied by the other member. After-wards, the evaluator could decide to modify his assessment or otherwise. The final score of each member will be the average of all the valuations, weighted by the weight of each criterion (i.e. $1/3$). Then, the $100$ points would be proportionally distributed to meet with the assessment criteria of the module. 


\subsection{Decision Making, Agreements and Conflict Resolution Methodology}

Making agile agreements and rapidly resolving conflicts are key factors towards maximizing our team output. Our methodology is inspired by the agents and multiagents theory ~\cite{BDI}, specifically in a practical reasoning agent BDI (Belief – Desires - Intentions).  In this manner, every team member possesses a valid point of view. Any difference between opinions goes through a deliberation process (where personal conflict is a conciliation process, which is a special type of deliberation), that would transform a set of options or desires to a set of intentions (this implies that each time there is a deliberation process, the first step is to state a set of options). Then we will plan and assign resources. However, the deliberation process required a balance due to time restrictions (based on our timetable and curve advance). This implies that we must state a mechanism to accelerate the process wherever it is appropriate (deliberation process output is the ideal, but practically we would not always have time for a full consensus). Also notice that we can modify our plan due a belief update. 
Each time we are unable to reach consensus due to time restrictions, we have specified an agreement protocol that precipitates a final decision where the group will move forward. For this, we use the \footnote{Borda Count: Each option receives $(m-1)$ points for each voter who vote as a first choice, $(m-2)$ for each voter who vote as a second choice, and so on, where $m$ is the number of options.  The winner option is that with more points. If there is a tie the decision will be randomly.} Borda Count vote procedure to try to maximize the preferences of the entire group.


Finally, for a personal conflict (such as a communication problem), there will be an arbitration protocol to resolve it. Basically, the team identifies those members that are not involved in the conflict and implore them to mediate. If after a conciliation process there is no agreement, referees will make the decision (only if there are 2 or more referees). If a referee abstains, then we would activate the vote procedure. The following flowchart summarizes our methodology.




\begin{figure}[ht]
\centering
\includegraphics[width=0.6\textwidth]{figs/met}
	\caption{Decision Making and Conflicts Resolution Methodology}
	\label{fig:decisions}
\end{figure}



%DEA~\cite{charnes1978measuring} is a non-parametric method that determines efficiency estimates for a set of similar entities, called decision making units (DMU). 
%We refer to the set of DMUs as ${\cal K}:=\{\DMU_1, \DMU_2,\dots,\DMU_K\}$, where ${
%K = |\cK|}$. 
