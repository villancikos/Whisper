Chat system applications is a very famous development in software engineering. Popular examples, just to mention a few, are chats like Whatsapp, Slack, Telegram, Skype, Snapchat, Facebook Messenger, among many others.

About the selected platforms, Stack Overflow Developer Survey in 2017 ~\cite{DeveloperStackoverflow} establishes that the most popular technologies are Android, iOS and Web. In addition, the market share of iOS and Android in 2016 is about 12.9 and 86,2 percent, respectively ~\cite{DeveloperStackoverflow}.  We also added the Web Client because web development has always been attractive, we believe that's the old big companies like Google, Amazon and Facebook started that way. Also, the entry barrier is lower than other tool-kits.

One of the most important reviews we research at the beginning, was proper methodologies to work as a group. Specifically, the agile methodology, proposed by Alistain Cockburn and Jim Highsmith ~\cite{rury346} was our first approach. However, the lack of experience set in and we didn't know how to tackle in advance many of the software challenges laying ahead. In addition, the proper division of each single task of our software made us struggle at some point, and we had to return to the foundations of the Agile methodology. That is, using an hybrid approach on which some tests were conducted at the end and others in each single iteration. In other words, our main challenge here, was deciding how to tackle each part of the overall system and make the integration tests. 

As part of the creation of this project, we ultimately decided that in order to get the most out of it, we needed to learn more about different tools. We wanted to pick a challenging technology with the robustness needed but also with vasts amounts of information on the Internet and in books to try and reduce the learning curve the most we could. 

For the clients we decided to go with Android, iOS and Web and Firebase as our back-end because we were thinking of focusing a lot on the User Experience inside the clients. We wanted users to have a good experience using our service and convince them to use it. From the beginning, we knew that the Authentication is a strong feature in Firebase. It allows you to control who can read your data and let you pick different services such as Facebook, Gmail, GitHub, and others to register and login. One of the main things we found that Firebase has is the real time database. Thanks to this feature, we could see all messages in real time in all the clients. 

As we mentioned before, Firebase was the best choice for our intended purposes, not only because is backed by Google. Aside all the benefits such as Web Analytics, Push Notifications, File storage, and the main reason, the price (which is free for our needs). Firebase gives you the flexibility to do a lot of ``back-end" code in the client side. Even though Firebase leverages a lot to the client, we realised it was interesting to test new ways of developing applications, as we only knew the basics around Object Oriented Programming. It gave us the flexibility to focus on the front-end development but also controlling features you normally do on a back-end. Another of the main reasons why we choose is because it has a really well written documentation on its website. Lastly, it takes minimum setup to begin developing, which was a key point for us as we wanted to finish the project on time while competing between studies and time. Even with this concern in mind we had some challenges finishing it but we learnt a lot on the process.
